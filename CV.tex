\documentclass[9pts,a4paper,sans]{moderncv}
\usepackage[utf8]{inputenc}
\usepackage[scale=0.85]{geometry}
\usepackage{helvet}
\usepackage[french]{babel}
\moderncvicons{awesome}
\nopagenumbers{}
\moderncvtheme[blue]{classic}
\firstname{Joris}
\photo[80pt]{../IMG_0490.jpg}
\familyname{OFFOUGA}
\title{Apprenti Ingénieur Linux Embarqué }
\address{103 Avenue Emile Counord}{33300 Bordeaux}{France}
\mobile{06 50 40 71 13}
\email{offougajoris@gmail.com}
\social[linkedin]{joris-offouga}
\social[github]{jorisoffouga}
\social[twitter]{jorisoffouga}
\extrainfo {\\}
\extrainfo {21 ans\\Pas d'enfants\\Nationalité: Gabonaise}
\quote{<<Passionné par le linux embarqué \& le developpement logiciel
, j'aimerai mettre en pratique mes connaissances dans le monde de l'entreprise pour les approfondir>>}

\begin{document}
\maketitle

\section{Expériences Professionnelles}
\cventry{} {Juillet 2018-Aujourd'hui} {Alternance} {HU\&CO} {Saint Jean D'Illac} {
	Développement et intregation d'applications sur un environnement linux embarqué d'un système d'alarme.
	\begin{itemize}
		\item Développement d'une Distribution Linux Embarqué sous Yocto/Openembedded et d'un Bootloader U-Boot sur cible NXP LS1012
		\item Développement d'un système de mise à jour pour les periphériques
		\item Gestion mis à jour sur cible NXP LS1012 via SWUpdate 
		\item Développement User Space:
		\begin{itemize}
			\item I$^{2}$C
			\item Applications microservices en Python 3.7
			\item Développement serveur en Django
		\end{itemize}
		\item Développement du firmware des periphériques sur cible T.I CC1310
		\item Mis en place de CI pour les periphériques sous Gitlab-CI 
		\item Outils: C, Python3.7, script shell, systemd 
	\end{itemize}
	Développement et intregration du firmware de drapeau/brassard connectés pour arbitres.
	\begin{itemize}
		\item Développement du firmware de drapeau/brassard
		\item Développement d'un outil de flashage de production pour les kits
	\end{itemize}
}

\cventry{} {Juin 2017-Juillet 2017} {Stage Technique} {ESTEI} {Bordeaux (Gironde)} {Aide à gestion du laboratoire d’électronique.} {}
{} {}

\section{Formation}
\cventry{} {2015-2020} {ESTEI} {Bordeaux} {Master en Systèmes Embarqués} {}

\section{Compétences}
\cvitem {Langages:} {C/C++, Python (3.7), Shells Unix, QT Framework} 
\cvitem {SOC:} {Broadcom BCM2837/35, NXP i.MX7, NXP LS1012, ST STM32MP1}
\cvitem {Linux Embarqué:}{ Développement:
	\begin{itemize}
		\item Bootloader : Das U-boot
		\item BuildSystem : Yocto/Openembedded, Buildroot
		\item Kernel: Noyaux GNU/Linux
		\item Mise à jour OTA : Mender, Rauc, SWUpdate	
	\end{itemize}
} 

\cvitem {MCU:} {HCS08(Freescale), ARM Cortex M0/M3 (Texas Instruments CC1310), ARM Cortex M4 (ST Microelectronics STM32F4)}
\cvitem {Protocoles:} {SPI, I$^{2}$C, UART, MAC TI-15.4, MQTT} 
\cvitem {Outils:} {Visual Studio Code, Eclipse, git, github, gitlab }
\cvitem {OS Temp Réel:} {FreeRTOS, Zephyr, TI-RTOS}

\section{Projets \& Logiciels Libres}
\cventry{} {Contributeur} {meta-openembedded} {} {\href{https://github.com/openembedded/meta-openembedded/commits?author=jorisoffouga}{Lien}} {}
\cventry{} {Contributeur} {meta-freescale} {} {\href{https://github.com/Freescale/meta-freescale/commits?author=jorisoffouga}{Lien}} {}
\cventry{} {Contributeur} {meta-freescale-3rdparty} {} {\href{https://github.com/Freescale/meta-freescale-3rdparty/commits?author=jorisoffouga}{Lien}} {}
\cventry{} {Contributeur} {meta-mender-community} {} {\href{https://github.com/mendersoftware/meta-mender-community/commits?author=jorisoffouga}{Lien}} {}
\cventry{} {Contributeur} {U-Boot} {} {\href{https://github.com/trini/u-boot/commits?author=jorisoffouga}{Lien}} {}
\cventry{} {Contributeur} {Buildroot} {} {\href{https://github.com/buildroot/buildroot/commits?author=jorisoffouga}{Lien}} {}
\cventry{} {Contributeur} {Zephyr} {} {\href{https://github.com/zephyrproject-rtos/zephyr/commits?author=jorisoffouga}{Lien}}{}
\cventry{} {Mainteneur} {meta-stm32mp1} {} {\href{https://github.com/bdx-iot/meta-stm32mp1/commits?author=jorisoffouga}{Lien}} {}
\cventry{} {Mainteneur} {meta-swupdate-dev} {} {\href{https://github.com/bdx-iot/meta-swupdate-dev/commits?author=jorisoffouga}{Lien}} {}

\section{Langues}
\cvline {Français} {Langue maternelle}
\cvline {Anglais} {Technique}
 
\section{Divers}
\cvline {Loisir} {Cinéma, Jeux-Vidéo, Musique, Programmation, Linux Embarqué}

\end{document}