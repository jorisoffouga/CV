\documentclass[10pts,a4paper,sans]{moderncv}
\usepackage[utf8]{inputenc}
\usepackage[scale=0.85]{geometry}
\usepackage{helvet}
\usepackage[french]{babel}
\moderncvicons{awesome}
\nopagenumbers{}
\moderncvtheme[blue]{classic}
\firstname{Joris}
\photo[80pt]{IMG_0490.jpg}
\familyname{OFFOUGA}
\title{Apprenti Ingénieur Systèmes Embarqués }
\address{103 Avenue Emile Counord}{33300 Bordeaux}{France}
\mobile{06 50 40 71 13}
\email{offougajoris@gmail.com}
\social[linkedin]{joris-offouga}
\social[github]{jorisoffouga}
\extrainfo {\\}
\extrainfo {20 ans\\Pas d'enfants\\Nationalité: Gabonaise}
\quote{<<Curieux, s\'{e}rieux et travailleur,j’ai une grande passion pour le d\'{e}veloppement logiciel et
pour l'IoT,\newline {} j'aimerai mettre en pratique mes connaissances dans le monde de l'entreprise
pour les approfondir>>}

\begin{document}
\maketitle

\section{Exp\'{e}riences Professionnelle}
\cventry{} {Juin 2017-Juillet 2017} {Stage Technique} {ESTEI} {Bordeaux (Gironde)} {Aide à gestion du laboratoire d’électronique.} {}
\cventry{} {Janvier 2017-Juin 2017} {Projet technique} {ESTEI} {Bordeaux(Gironde)} {Mise en œuvre d'actionneur du véhicule et d'une interface homme-machine à travers une télécommande sans fil .\href{https://github.com/jorisoffouga/projet_vehicule_interactif_B2}{Lien}}
{} {}

\section{Formation}
\cventry{} {2017-2018} {ESTEI} {Bordeaux} {Bachelor Systèmes Embarqués \& Robotique} {}
\cventry{} {2015-2017} {ESTEI} {Bordeaux} {Validation d'une classe préparatoire en Systèmes Embarqués \& Robotique} {}
\cventry{} {2014-2015} {Libreville, Gabon} {Baccalauréat Scientifique} {} {}

\section{Compétences}
\cvdoubleitem {Langages:} {C \newline {} Shells Unix \newline {} Assembleur CISC \newline {} \LaTeX} {MCU:} {Fresscale(HCS08) \newline {} ARM Cortex M4(STM32)}
\cvdoubleitem {SOC:} {Broadcom BCM2837} {Outils:} {Altium Designer \newline {} Visual Studio Code \newline {} Eclipse}
\cvdoubleitem {Protocoles:} {SPI, I2C, UART} {Linux Embarqu\'{e}:}{ 
	Toolchain : CrossTool-NG
	\newline {} Bootloader : u-boot
	\newline {} Build System : Yocto Project (Niveau débutant)
	\newline {} Noyaux GNU/Linux
}
\cvitem {Outils GNU Linux} {make, Cmake}

\section{Langues}
\cvline {Français} {Langue maternelle}
\cvline {Anglais} {Technique}
 
\newpage
\section{Divers}
\cvline {Loisir} {Cinéma, Jeux-Vidéo, Musique, Programmation, Linux Embarqué}

\end{document}